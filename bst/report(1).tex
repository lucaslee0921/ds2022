\documentclass[UTF8]{ctexart}
\documentclass[a4paper]{article}

\usepackage{listings,xcolor}
\usepackage{graphicx}
\usepackage[a4paper,left=25.4mm,right=25.4mm,top=29.8mm,bottom=29.8mm]{geometry}

\lstset{numbers=left,
    commentstyle=\color{blue!50},
    columns=flexible}

\title{作业二:BST 排序算法的实现}

\author{李巧杰 \\ 3210300366 \\ 信息与计算科学}

\begin{document}

\maketitle

二叉搜索书的定义

\begin{flushright}
——摘自百度百科《XXX》
\end{flushright}

\section{设计思路}

\阅读并理解课本所提供的头文件 dsexceptions.h 里头编写的所有代码
\hphantom 空例如:class UnderflowException、IllegalArgumentException 等

\编写测试程序 main.cpp 并适当地理解课本注释以方便理解代
\hphantom 空例如:Node结构中包含数据以及两个指针

\编写首先先判断 _mode是否为一或零。若是为一将透过random shuffle来打乱数组顺序,才进行从新排序。反之,将不打乱顺序直接顺序排列
\在判断之后呢,进行构建一个二叉树以存放数组,并且假设左分支总是比右分支及母节点还要小,这样就可以确保数字是从小到大的被读出。

\end{lstlisting}

\end{flushleft}

\end{document}
