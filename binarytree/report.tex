\documentclass[UTF8]{ctexart}
\documentclass[a4paper]{article}

\usepackage{listings,xcolor}
\usepackage{graphicx}
\usepackage[a4paper,left=25.4mm,right=25.4mm,top=29.8mm,bottom=29.8mm]{geometry}

\lstset{numbers=left,
    commentstyle=\color{blue!50},
    columns=flexible}

\title{作业 Binary Tree}

\author{李巧杰 \\ 3210300366 \\ 信息与计算科学}

\begin{document}

\maketitle

\begin{center}
二叉树的定义
\end{center}

\   二叉树(Binary tree)是树形结构的一个重要类型。许多实际问题抽象出来的数据结构往往是二叉树形式,即使是一般的树也能简单地转换为二叉树,而且二叉树的存储结构及其算法都较为简单,因此二叉树显得特别重要。二叉树特点是每个节点最多只能有两棵子树,且有左右之分 [1]  。
二叉树是n个有限元素的集合,该集合或者为空、或者由一个称为根(root)的元素及两个不相交的、被分别称为左子树和右子树的二叉树组成,是有序树。当集合为空时,称该二叉树为空二叉树。在二叉树中,一个元素也称作一个节点

\begin{flushright}
——摘自百度百科《二叉树》
\end{flushright}

\section{设计思路}

\subsection{了解二叉树的原理,并起草设计二叉树的类}
\hphantom 空例如: 了解二叉树与Node之间的关系等等

\subsection{根据其原理设计一个二叉树与Node之间的类}
\hphantom 空例如:用代码表达作者对各数据结构之间的关系

\subsection{编写测试程序 binarytree.h 用作检验代码的可行性等等}
\hphantom 空例如:输出运行程序所需要的时间

\end{document}





