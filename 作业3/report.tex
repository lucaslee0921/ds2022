\documentclass[UTF8]{ctexart}
\documentclass[a4paper]{article}

\usepackage{listings,xcolor}
\usepackage{graphicx}
\usepackage[a4paper,left=25.4mm,right=25.4mm,top=29.8mm,bottom=29.8mm]{geometry}

\lstset{numbers=left,
    commentstyle=\color{blue!50},
    columns=flexible}

\title{作业三:AvlTree 课后作业4.37}

\author{李巧杰 \\ 3210300366 \\ 信息与计算科学}

\begin{document}

\maketitle

\begin{center}
Avl树的定义
\end{center}

\   在计算机科学中,AVL树是最先发明的自平衡二叉查找树。在AVL树中任何节点的两个子树的高度最大差别为1,所以它也被称为高度平衡树。增加和删除可能需要通过一次或多次树旋转来重新平衡这个树。AVL树得名于它的发明者G. M. Adelson-Velsky和E. M. Landis,他们在1962年的论文《An algorithm for the organization of information》中发表了它

\begin{flushright}
——摘自百度百科《Avl树》
\end{flushright}

\section{设计思路}

\subsection{阅读并理解课本所提供的头文件 AvlTree.h 里头编写的所有代码}
\hphantom 空例如: 其功能函数及各种已被定义的变量等等

\subsection{根据题目所要求的 设计一个函数用作input一个二叉树,T,k1,k2,并符合题目要求}
\hphantom 空例如:设计了一个符合题目的函数在文件AvlTree.h

\subsection{编写测试程序 main.cpp 用作检验代码的复杂性,用时等等}
\hphantom 空例如:输出运行程序所需要的时间

\subsection{测试整个程序,记录结果,是否符合题意}

\   首先,要求该程序应该循环多少次才能符合到测试的要求,因此主函数为设立循环次数;其次设计一个void程序,先创建一个Avl树,且将1-N的数字输入进去Avl树,并且设置上下确界以确保所有的值都在这个区间里。

\   其次,程序需要记录循环开始时间、结束时间、记录整个程序一次循环所需要使用的时间,以用作反映其复杂性

\newpage

\section{理论分析}

\subsection{论证复杂性符合要求:}

\   由于Avl树是属于一维范围内的搜索程序,因此在执行程序中遍历的时间是 O(K)。如果在某种情况下,找到了大量的节点,那一定是与树的深度成正比。为此,由于平均深度为O(log N ),则我们可以得到平均界限为O(K+log N )

\end{document}





